\resetproblem \begingroup % \jeolmheader
    \def\jeolmdate{8 июня 2018 г., пара 1}%
    \def\jeolmgroupname{Группа 9-1}%
    \def\jeolmauthors{Кушнир А., Тихонов~Ю.}%
\jeolmheader \endgroup

\worksheet*{Дополнительные задачи по комбинаторике}

\begin{problems}

\item Последовательность $a_n$ натуральных чисел называется \emph{полиномиальной}, если существует многочлен $P(x)$ с целыми коэффициентами такой, что $a_n = P(n)$ при всех натуральных $n$. Можно ли раскрасить натуральные числа в два цвета так, чтобы не было одноцветных полиномиальных последовательностей?

\item Стрелки полного ориентированного графа раскрашены в два цвета. Докажите, что в нём существует вершина, от которой можно добраться до любой другой вершины по некоторому монохромному пути одного из двух цветов.

\item В $100$ ящиках лежат яблоки, апельсины и бананы. Докажите, что можно так выбрать $51$ ящик, что в них окажется не менее половины всех яблок, не менее половины всех апельсинов и не менее половины всех бананов.

\end{problems}