\resetproblem \begingroup % \jeolmheader
    \def\jeolmdate{2 июня 2018 г., пара 2}%
    \def\jeolmgroupname{Группа 9-3}%
    \def\jeolmauthors{Афризонов Д., Тихонов~Ю.}%
\jeolmheader \endgroup

\worksheet*{Ориентированные графы, дополнительные задачи}

% По плану на втором занятии группы 9-3 дорешивание первого листик + разбор
% на второй половине занятия.
% Эту добавку надо давать ТОЛЬКО в случае, когда школьник решил ВЕСЬ первый
% листик в самой паре.
% Если таких не нашлось, то не давать вообще.
% У других групп на втором занятии, вероятно, появится новый листик.

% Про разбор.
% Программа минимум — задачи 2 и 3.
% Постараться еще в ходе разбора отложить у них в голове структуру компонент
% сильной связности.
% При разборе задачи 3 надо продемонстрировать тем, кто не понял и не верит,
% в какой степени при выделении компонент задачка может рассыпаться.
% По остальным задачам не лишним будет намекнуть, что произойдет то же самое.

\setproblem{6}

\begin{problems}

\item
Докажите, что в~полном ориентированном сильно связном графе на~$n \geq 3$
вершинах через каждую его вершину проходит
\\
\subproblem простой цикл длины $3$;
\\
\subproblem простой цикл любой длины $k$, где $3 \leq k \leq n$.

\item
Докажите, что в~полном ориентированном графе на~$n \geq 7$ вершинах всегда
найдется вершина, инвертированием всех стрелок в~которой можно добиться того,
чтобы граф стал сильно связным.

\end{problems}

