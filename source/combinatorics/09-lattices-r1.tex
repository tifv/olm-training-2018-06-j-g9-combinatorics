\resetproblem \begingroup % \jeolmheader
    \def\jeolmdate{9 июня 2018 г., пара 1}%
    \def\jeolmgroupname{Группа 9-1}%
    \def\jeolmauthors{Афризонов Д., Кушнир А., Тихонов~Ю.}%
\jeolmheader \endgroup

\worksheet*{Решётки}

\begin{problems}

\item
В вершинах некоторого квадрата сидит по одному кузнечику. Каждую минуту ровно один кузнечик перепрыгивает центрально симметрично через какого-то другого.
\\
\subproblem Докажите, что если кузнечики вновь оказались в вершинах изначального квадрата, то каждый вернулся именно на свою стартовую позицию.
\\
\subproblem Могут ли кузнечики когда-нибудь оказаться в вершинах большего квадрата, чем изначальный?
% \item В трёх вершинах квадрата сидят кузнечики. Каждый год один из кузнечиков перепрыгивает через одного из двух других (т.\,е. отражается центрально симметрично). В конце они вновь оказываются в каких-то трёх вершинах исходного квадрата. Докажите, что каждый кузнечик сидит в своей стартовой вершине.

% \item Дан выпуклый $31500$-угольник, вершины которого располагаются в узлах целочисленной решётки. Докажите, что у него найдётся сторона длины по крайней мере $98$.

\item На координатной плоскости расположена фигура площади больше $1$. Докажите, что в ней можно отметить две различные точки так, чтобы вектор, их соединяющий, имел целые координаты.

\item На плоскости расположено несколько квадратов со стороной $1$, параллельных осям координат. Докажите, что в плоскость можно вбить несколько гвоздей так, чтобы каждый квадрат был прибит ровно одним гвоздем.

% \item Два вектора $\mathbf{u}$ и $\mathbf{v}$ с целыми координатами таковы, что все целые точки плоскости можно выразить их целочисленными линейными комбинациями (т.\,е. выражениями вида $n \mathbf{u} + m \mathbf{v}$, где $m$ и $n$ --- целые числа). Докажите, что параллелограмм, натянутый на $\mathbf{u}$ и $\mathbf{v}$, имеет площадь $1$.

\item Внутри правильного треугольника отмечена точка. Её отразили несколько раз относительно сторон треугольника (в случайном порядке), и она опять попала внутрь треугольника. Докажите, что она вернулась на исходное место.

\item Существует ли на клетчатой плоскости замкнутая ломаная с нечётным числом звеньев одинаковой длины, все вершины которой лежат в узлах целочисленной решётки?

\item Время жизни марсиан составляет в точности $100$ лет. Каждый марсианин рождается ровно в момент начала какого-то года. За всё время жизни вселенной существовало нечётное число марсиан. Докажите, что можно выделить по крайней мере $100$ лет (не обязательно подряд идущих), когда жило нечётное число марсиан.

% \item (\emph{Теорема Минковского}) На координатной плоскости дана выпуклая ограниченная фигура площади больше, чем $4$, симметричная относительно начала координат. Докажите, что она содержит целую точку, отличную от начала координат.
 % Фигура на плоскости называется \emph{выпуклой}, если вместе с каждой парой своих точек $A$ и $B$ она целиком содержит отрезок $AB$.

% \item На плоскости расположено $2015$ квадратов со стороной $1$, параллельных осям координат. Докажите, что множество точек плоскости, покрытых нечётным количеством таких квадратов, имеет площадь не менее $1$.

\item На клетчатой плоскости дан многоугольник, вершины которого лежат в узлах целочисленной решётки. Каждую клетку плоскости разрезали красными диагоналями. В результате плоскость оказалась разбита на квадратики с красной границей, которые покрасили в шахматном порядке в чёрный и белый цвет. Докажите, что площадь белой части исходного многоугольника равна площади чёрной части.

\item Рассмотрим на плоскости множество точек с вещественными координатами $(x, y)$ таких, что неравенство $mx + ny \geqslant \frac{1}{2} \cdot (m^2 + n^2)$ имеет ровно $2018$ целочисленных решений $(m, n)$. Найдите площадь этого множества.

% \item На координатной плоскости отмечены некоторые целые точки. Известно, что внутри любого круга радиуса $2015$ хотя бы одна точка отмечена. Докажите, что существует окружность, проходящая хотя бы через четыре отмеченные точки.

\end{problems}