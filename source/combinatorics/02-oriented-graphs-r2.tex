\resetproblem \begingroup % \jeolmheader
    \def\jeolmdate{2 июня 2018 г., пара 1}%
    \def\jeolmgroupname{Группа 9-2}%
    \def\jeolmauthors{Афризонов Д., Коваленко К., Тихонов~Ю.}%
\jeolmheader \endgroup

\worksheet*{Ориентированные графы}

% Мне кажется, что тема занятия для второй группы понятие класса
% эквивалентности должно быть новым. По крайней мере на кружке по геометрии с
% векторами была беда совсем ровно потому, что они не понимали, что это такое.
% Лучше им сначала рассказать, и только потом дать листик.

% В начале занятия предполагается рассказ про отношение эквивалентности:
% определение, разбиение на классы эквивалентности + несколько примеров (кольцо
% остатков по модулю, векторы, компоненты связности). Далее нужно дать
% конструкцию компонент сильной связности как классов эквивалентности
% по отношению обоюдной достижимости на вершинах.

\emph{Ориентированный граф}~--- конечное множество вершин, некоторые из~которых
соединены стрелками.
В~ориентированном графе запрещены кратные стрелки (даже в~разных направлениях)
и~петли, если не~оговорено иное.
Ориентированный граф \emph{полный}, если любая пара его вершин соединена
единственной стрелкой.
Ориентированный граф называется \emph{сильно связным}, если из~любой его
вершины можно добраться до~любой другой по~стрелкам.

\begin{problems}

\item
Обозначим через $A_{1}$, $A_{2}$, \ldots, $A_{k}$ компоненты сильной связности
ориентированного графа~$G$.
Введём вспомогательный ориентированный граф~$H$ с~вершинами
$a_{1}$, $a_{2}$, \ldots, $a_{k}$;
проведём стрелку из~$a_{i}$ в~$a_{j}$ в~графе~$H$, если существует хотя бы одна
стрелка из~$A_{i}$ в~$A_{j}$ в~графе~$G$.
\\
\subproblem
Докажите, что в~графе~$H$ нет циклов и кратных стрелок.
\\
\subproblem
Докажите, что если $G$ был полным ориентированным графом, то вершины графа~$H$
можно пронумеровать так, чтобы все стрелки в~графе~$H$ вели из вершины
с~меньшим номером в~вершину с~б\'{о}льшим.
\\
Граф~$H$ называется \emph{конденсацией} ориентированного графа~$G$.

\item
Любые два города Табулистана соединены дорогой с~односторонним движением.
В~стране транспортные проблемы: нет сильной связности.
Докажите, что можно раздать все города республиканцам и~демократам так, чтобы
обе партии получили хотя~бы по~одному городу и~чтобы никакая дорога не~вела
из~города республиканцев в~город демократов.

\item
В~условиях предыдущей задачи, если городов хотя бы три, на~каком наименьшем
числе дорог президенту Табулистана надо изменить направление движения, чтобы
решить транспортные проблемы and make Tabulistan great again?

\item
Докажите, что в~полном ориентированном сильно связном графе на~$n \geq 3$
вершинах через каждую его вершину проходит
\\
\subproblem простой цикл длины $3$;
\\
\subproblem простой цикл любой длины $k$, где $3 \leq k \leq n$.

\item
В~стране 1001 город, любые два города соединены дорогой с~односторонним
движением.
Из~каждого города выходит ровно 500 дорог, в~каждый город входит ровно 500
дорог.
От~страны отделилась независимая республика, в~которую вошли 668 городов.
Докажите, что из~любого города этой республики можно добраться до~любого
другого, не~выезжая за~пределы республики.

\item
Докажите, что в~полном ориентированном графе на~$n \geq 7$ вершинах всегда
найдётся вершина, инвертированием всех стрелок в~которой можно добиться того,
чтобы граф стал сильно связным.

\item
Докажите, что в~полном ориентированном графе на~$n$ вершинах ($n \geq 6$)
существует такой гамильтонов путь $v_1 \to v_2 \to v_3 \to \ldots \to v_{n}$,
что $v_{1} \to v_{n}$, если
\\
\subproblem $n$~--- чётное;
\qquad
\subproblem $n$~--- нечётное.

% С трен. олимпиады весны-2017.

% Про 2n - 3 и про гамильтонов цикл давал уже.

\end{problems}

