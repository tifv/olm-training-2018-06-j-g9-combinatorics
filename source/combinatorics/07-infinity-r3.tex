\resetproblem \begingroup % \jeolmheader
    \def\jeolmdate{7 июня 2018 г., пара 1}%
    \def\jeolmgroupname{Гексаэдры}% 9-3
    \def\jeolmauthors{Коваленко К., Кушнир А., Тихонов Ю.}%
\jeolmheader \endgroup

\worksheet*{Сколь угодно много и бесконечно много}

\begin{problems}

\item Натуральные числа раскрасили в два цвета. Обязательно ли существует одноцветная бесконечная арифметическая прогрессия? 

\item Зафиксировано натуральное число $n$. Будем говорить, что числовая строка $(x_1, x_2, \ldots, x_n)$ \emph{лексикографически больше} числовой строки $(y_1, y_2, \ldots, y_n)$, если есть такой $i \in \{ 1, 2, \ldots, n \}$, что $x_1 = y_1, x_2 = y_2, \ldots, x_{i - 1} = y_{i - 1}, x_i > y_i$.\\
\claim{Лемма} Докажите, что не существует бесконечной лексикографически убывающей последовательности строк длины $n$, состоящих из натуральных чисел.

\item В Табулистане все купюры местной валюты обладают целочисленными номиналом. В местном банке готовы принять любую купюру и выдать вместо неё любое количество купюр меньшего номинала. У барона Мюнхгаузена есть купюра достоинством $100$, и он утверждает, что сможет жить на эти деньги бесконечно долго, расходуя при этом не менее $1$ единицы валюты в день. Стоит ли ему доверять?

\item \claim{Лемма Кёнига} Докажите, что в бесконечном дереве, степень каждой вершины которого конечна, найдётся бесконечный простой путь.

\item В стране роботов любые конечные или бесконечные последовательности из $0$~и~$1$ называются \emph{словами}. Некоторые конечные слова объявлены \emph{матерными}. Слово называется \emph{цензурным}, если оно не содержит матерных подслов.\\
\subproblem Известно, что существуют сколь угодно длинные конечные цензурные слова. Докажите, что существует бесконечное цензурное слово.\\
\subproblem Известно, что матерных слов конечное число и существует бесконечное цензурное слово. Докажите, что существует бесконечное периодическое цензурное слово.
%\subproblem Останется ли верным утверждение предыдущего пункта, если снять ограничение на конечность множества матерных слов?

\item Допустим, что любую конечную карту можно правильным образом раскрасить в 4 цвета. Докажите, что тогда и бесконечную карту тоже можно раскрасить в 4 цвета.

\item Вася закрашивает точки координатной плоскости с натуральными координатами. За одну операцию он выбирает ешё не закрашенную точку ($m, n$) и закрашивает все точки $(x, y)$, у которых $x \geq m$ и $y \geq n$. Может ли Вася проворачивать свои операции бесконечно долго?

\item Сборная Табулистана с $2000$ года ежегодно участвует в финале всероссийской олимпиады школьников по математике, которая проводится по параллелям 9,~10,~11 класса. Год $n$ называется \emph{провальным}, если для каждого предыдущего года $m$ существует параллель, по которой в год $n$ сборная Табулистана взяла меньше дипломов, чем в год $m$. Могут ли все годы участия, начиная с $2001$, быть провальными?

% \item Есть несколько эталонных квадратов $1 \times 1$, стороны которых раскрашены в какие-то цвета и параллельны осям координат. Разрешается полученные параллельным переносом копии этих квадратов выкладывать на плоскость, причём соседние квадраты должны задавать одинаковую раскраску их общей стороны. Известно, что исходный набор квадратиков таков, из его копий можно выложить сколько угодно большой квадрат $n \times n$. Докажите, что копиями квадратов набора можно замостить всю плоскость.
% Оставим на повторение.

% \item Все натуральные числа покрасили в~несколько цветов. Докажите, что найдется цвет такой, что для любого натурального~$n$ бесконечно много чисел этого цвета делится на~$n$. 
% Не совсем в тему; простой конструктив. Перемножим все плохие числа для всех цветов.

%\item Натуральные числа покрашены в~синий и~зеленый цвета, причем чисел каждого цвета бесконечно много. Докажите, что можно выбрать миллион синих и~миллион зеленых чисел так, что сумма выбранных синих равна сумме выбранных зеленых. 

%\item Имеется бесконечная шахматная доска. Из~нее выкинули все клетки, у~которых обе координаты отрицательны или обе координаты больше тысячи. Можно~ли обойти такую фигуру (побывав в~каждой клетке ровно по~одному разу) конем?
% Вообще не в тему. Асимптотика.

% \item Каждая точка плоскости, имеющая целочисленные координаты, раскрашена в~один из~$n$ цветов. Докажите, что найдется прямоугольник с~вершинами в~точках одного цвета. 

% \item Каждая точка плоскости, имеющая целочисленные координаты, раскрашена в один из $n$~цветов. Докажите, что для любого $N$ найдется цвет~$X$ такой, что найдется точка цвета~$X$, с которой на одной горизонтали бесконечно много точек цвета~$X$, а на одной вертикали не меньше $N$ точек цвета~$X$. 
% Не нравится она мне.

% \ii За~дядькой Черномором выстроилось чередой бесконечное число богатырей. Докажите, что он может приказать части из~них выйти из~строя так, чтобы в~строю осталось бесконечно много богатырей и~все они стояли по~росту (не~обязательно в~порядке убывания роста). 

% \item Имеется возрастающая последовательность натуральных чисел $a_{1} < a_{2} < \ldots < a_{n} < \ldots$, где для всякого $k$ выполнено $a_{k+1} - a_{k} < 10^6$. Докажите, что найдутся различные $i$ и~$j$ такие, что $a_i$ делится на~$a_j$. 
% Крутая задача, но не совсем по теме.

%\item Натуральные числа покрасили в~$100$ цветов. Докажите, что найдется цвет~$X$ такой, что есть бесконечно много пар чисел цвета~$X$ на~расстоянии меньше $200$.
% Следующая --- примерно про то же.

% \item Натуральные числа покрашены в~несколько цветов. Докажите, что найдутся цвет~$X$ и~число~$m$ такие, что для любого натурального $k$ существуют натуральные числа $a_{1} < a_{2} < \ldots < a_{k}$ цвета~$X$, где $a_{i+1} - a_{i} < m$ при всех $1 \leq i < k$.

% \item Мудрецы выстроены в бесконечный ряд. На каждом --- колпак одного из двух цветов. Каждый мудрец знает свой номер в ряду и видит всех мудрецов с большими номерами. Мудрецов по команде должны одновременно назвать предполагаемый цвет своего колпака. Докажите, что мудрецы могут заранее договориться о стратегии так, чтобы при любой расстановке колпаков не угадало цвет своего колпака лишь конечное число мудрецов.
% Задача от Саши Фёдорова.

\end{problems}