\resetproblem \begingroup % \jeolmheader
    \def\jeolmdate{1 июня 2018 г., пара 3}%
    \def\jeolmgroupname{Группа 9-3}%
    \def\jeolmauthors{Кушнир А., Тихонов~Ю.}%
\jeolmheader \endgroup

\worksheet*{Ориентированные графы}

% В начале занятия предполагается рассказ про отношение эквивалентности:
% определение, разбиение на классы эквивалентности + несколько примеров (кольцо
% остатков по модулю, векторы, компоненты связности). Далее нужно дать
% конструкцию компонент сильной связности как классов эквивалентности
% по отношению обоюдной достижимости на вершинах.

\emph{Ориентированный граф}~--- конечное множество вершин, некоторые из~которых
соединены стрелками.
В~ориентированном графе запрещены кратные стрелки (даже в~разных направлениях)
и~петли, если не~оговорено иное.
Ориентированный граф \emph{полный,} если любая пара его вершин соединена
единственной стрелкой.
Ориентированный граф называется \emph{сильно связным,} если из~любой его
вершины можно добраться до~любой другой по~стрелкам.

\begin{problems}

\item
Докажите, что в~любом полном ориентированном графе существует путь, проходящий
по~всем вершинам ровно по~одному разу \emph{(гамильтонов путь)}.

\item
Обозначим через $A_{1}$, $A_{2}$, \ldots, $A_{k}$ компоненты сильной связности
ориентированного графа~$G$.
Введём вспомогательный ориентированный граф~$H$ с~вершинами
$a_{1}$, $a_{2}$, \ldots, $a_{k}$;
проведём стрелку из~$a_{i}$ в~$a_{j}$ в~графе~$H$, если существует хотя бы одна
стрелка из~$A_{i}$ в~$A_{j}$ в~графе~$G$.
\\
\subproblem
Докажите, что в~графе~$H$ нет кратных стрелок (т.\,е. что между любыми двумя
компонентами графа~$G$ рёбра идут только в~одном направлении).
\\
\subproblem
Докажите, что в~графе~$H$ нет циклов.
\\
\subproblem
Докажите, что если $G$ был полным ориентированным графом, то вершины графа~$H$
можно пронумеровать так, чтобы все стрелки в~графе~$H$ вели из вершины
с~меньшим номером в~вершину с~б\'{о}льшим.
\par
Граф~$H$ называется \emph{конденсацией} ориентированного графа~$G$.

\item
Про ориентированный граф известно, что в~нём не~существует маршрута,
проходящего по~всем вершинами
(даже если в~некоторые вершины заходить по~нескольку раз).
Докажите, что вершины графа можно раскрасить в~красный и~синий цвет так, чтобы
оба цвета присутствовали и~чтобы никакая стрелка не~вела из~красной вершины
в~синюю.

\item
Даны $n$~точек, двое играют в~игру.
За~один ход игрок может соединить стрелкой две точки, не~соединённые ранее.
Запрещено оставлять после своего хода сильно связный граф.
Проигрывает тот, кто не~может сделать ход.
Изначально никаких стрелок не~было.
Кто выигрывает: начинающий или его соперник?

\item
Докажите, что в~любом сильно связном полном ориентированном графе на~$n \geq 3$
вершинах существует цикл, проходящий по~всем вершинам ровно по~одному разу
\emph{(гамильтонов цикл)}.

\item
В~стране $1001$ город, любые два города соединены дорогой с~односторонним
движением.
Из~каждого города выходит ровно $500$ дорог, в~каждый город входит ровно
$500$ дорог.
От~страны отделилась независимая республика, в~которую вошли $668$ городов.
Докажите, что из~любого города этой республики можно добраться до~любого
другого, не~выезжая за~пределы республики.

\end{problems}

