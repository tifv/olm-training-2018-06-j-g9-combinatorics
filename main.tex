\documentclass[a4paper,10pt]{article}
\usepackage[utf8]{inputenc}
\title{training-2018-06-j-g9-combinatorics}

% Конфигурация — общее {{{1

\usepackage{multicol}
\usepackage[svgnames]{xcolor}
\usepackage{graphicx}

\usepackage{jeolm}
\usepackage{jeolm-groups}

\usepackage[T2A]{fontenc}
\usepackage{anyfontsize}
\usepackage{amsmath}
\usepackage{amssymb}
\usepackage{upgreek}
\AtBeginDocument{\swapvar{phi}\swapvar{epsilon}}
\AtBeginDocument{\swapvar[up]{phi}\swapvar[up]{epsilon}}
\AtBeginDocument{\let\geq\geqslant\let\leq\leqslant}
\usepackage[russian]{babel}
\usepackage{parskip}
\pagestyle{empty}

% Конфигурация — страницы {{{1

\usepackage{geometry}
% Размер «логической» страницы
\geometry{a5paper,portrait,vmargin={2em,2em},hmargin={2em,2em}}
%\geometry{a6paper,landscape,vmargin={1.5em,1.5em},hmargin={1.5em,1.5em}}

\usepackage{pgfpages}
% Размещение «логических» страниц на «физической»
\pgfpagesuselayout{resize to}[a4paper]
%\pgfpagesuselayout{2 on 1}[a4paper,landscape]
%\pgfpagesuselayout{4 on 1}[a4paper,landscape]

% Конфигурация — локальное {{{1

\renewcommand\jeolminstitution
  {Московские сборы по математике}
\renewcommand\jeolmdaterange
  {1--14 июня 2018\,г.}

% Конфигурация — размер шрифта {{{1

% Уменьшить основной размер шрифта
%\AtBeginDocument{\fontsize{9.00}{10.80}\selectfont}
% (по умолчанию 10.00 и 12.00 соответственно)

% }}}1


\begin{document}

\clearpage\resetproblem \begingroup % \jeolmheader
    \def\jeolmdate{09 июня 2018 г., пара 2}%
    \def\jeolmgroupname{9-3}%
    \def\jeolmauthors{Соколов~А.\,А.}%
\jeolmheader \endgroup

\worksheet*{Теорема Паскаля}

\claim{Теорема Паскаля} Даны шесть точек $A, B, C, D, E, F$ на одной окружности. Тогда пересечения прямых $AB$ и $DE$, $BC$ и $EF$, $CD$ и $FA$ лежат на одной прямой.


\begin{problems}

\item В окружность вписан шестиугольник $ABCDEF$. Отрезок $AC$ пересекается с отрезком $BF$ в точке $X$, $BE$ с $AD$ -- в точке $Y$ и $CE$ и $DF$ в точке $Z$. Докажите, что точки $X$, $Y$ и $Z$ лежат на одной прямой\footnote{Попробуйте найти на картинке два подобных треугольника $ABY$ и $EDY$ с изогонально сопряжёнными точками $X$ и $Z$}.

%\item  \emph{(Теорема Паскаля для треугольника)}  Докажите, что в неравнобедренном треугольнике точки пересечения касательных к описанной окружности, восстановленных в вершинах, с противоположными сторонами лежат на одной прямой.
\item Окружность, проходящая через вершины $A$ и $D$ основания трапеции $ABCD$, пересекает боковые стороны $AB$, $CD$ в точках $P$, $Q$, а диагонали -- в точках $E, F$. Докажите, что прямые $BC, PQ, EF$ пересекаются в одной точке.

%\item Даны треугольник $ABC$ и некоторая точка $T$. Пусть $P$ и $Q$ -- основания перпендикуляров, опущенных из точки $T$ на прямые $AB$ и $AC$ соответственно, a $R$ и $S$ -- основания перпендикуляров, опущенных из точки $A$ на прямые $TC$ и $TB$ соответственно. Докажите, что точка пересечения прямых $PR$ и $QS$ лежит на прямой $BC$.

\item \emph{(Теорема Паскаля для четырёхугольника)} Докажите, что прямая, соединяющая точки пересечения пар противоположных сторон вписанного в окружность четырёхугольника, совпадает с прямой, соединяющей точки пересечения пар касательных к этой окружности, восставленных в противоположных вершинах.


\item Внутри треугольника $ABC$ отмечена точка $P$. Прямые $AP, BP, CP$ вторично пересекают описанную окружность треугольника $ABC$ в точках $A_1, B_1, C_1$ соответственно. Докажите, что главные диагонали шестиугольника, полученного пересечением треугольников $ABC$ и $A_1B_1C_1$ пересекаются в точке $P$.


\item Четырехугольник $ABCD$ вписан в окружность с центром $O$. Точка $X$ такова, что $\angle BAX = \angle CDX = 90^{\circ}$. Докажите, что точка пересечения диагоналей четырехугольника $ABCD$ лежит на прямой $XO$.



\item Пусть $A'$ -- точка, диаметрально противоположная точке $A$ в описанной окружности треугольника $ABC$ с центром $O$. Касательная к описанной окружности в точке $A'$ пересекает прямую $BC$ в точке $X$. Прямая $OX$ пересекает стороны $AB$ и $AC$ в точках $M$ и $N$. Докажите, что $OM = ON$.

\item На сторонах $AB$ и $AC$ остроугольного треугольника $ABC$ выбрали соответственно точки $M$ и $N$ так, что отрезок $MN$ проходит через центр $O$ описанной окружности треугольника $ABC$. Пусть $P$ и $Q$ -- середины отрезков $CM$ и $BN$. Докажите, что $\angle POQ = \angle BAC$.
% 3ка с летних сборов 2014

% \item Треугольники $ABC$ и $A'B'C'$ вписаны в одну и ту же окружность, и их пересечением является шестиугольник. Докажите, что главные диагонали шестиугольника пересекаются в одной точке.


%\item На стороне $AB$ треугольника $ABC$ взята точка $D$. В угол $ADC$ вписана окружность, касающаяся изнутри описанной окружности треугольника $ACD$, а в угол $BDC$ -- окружность, касающаяся изнутри описанной окружности треугольника $BCD$. Оказалось, что эти окружности касаются отрезка $CD$ в одной и той же точке $X$. Докажите, что перпендикуляр, опущенный из $X$ на $AB$, проходит через центр вписанной окружности треугольника $ABC$.




%\item В  треугольнике $ABC$ проведены  высоты $AA_1$ и $BB_1$ и  биссектрисы $AA_2$ и $BB_2$; вписанная  окружность  касается  сторон $BC$ и $AC$ в  точках $A_3$ и $B_3$.  Докажите,  чтопрямые $A_1B_1, A_2B_2, A_3B_3$ пересекаются  в  одной  точке или параллельны.

%\item  Равносторонний треугольник $ABC$ вписан в окружность $\Omega$ и описан вокруг окружности $\omega$. На сторонах $AC$ и $AB$ выбраны точки $P$ и $Q$ соответственно так, что отрезок $PQ$ проходит через центр треугольника $ABC$. Окружности $\Gamma_b$ и $\Gamma_c$ построены на отрезках $BP$ и $CQ$ как на диаметрах. Докажите, что окружности $\Gamma_b$ и $\Gamma_c$ пересекаются в двух точках, одна из которых лежит на $\Omega$, а другая -- на $\omega$.

\end{problems}

\end{document}

% Архив {{{1

\clearpage\resetproblem \begingroup % \jeolmheader
    \def\jeolmdate{1 июня 2018 г., пара 2}%
    \def\jeolmgroupname{Группа 9-1}%
    \def\jeolmauthors{Кушнир А., Тихонов~Ю.}%
\jeolmheader \endgroup

\worksheet*{Ориентированные графы}

% Первой группа в целом должна быть знакома с темой, и ей лучше дать листик
% сразу и анонсировать, что формат рассказа — ликбез.

% В начале занятия предполагается рассказ про отношение эквивалентности:
% определение, разбиение на классы эквивалентности + несколько примеров (кольцо
% остатков по модулю, векторы, компоненты связности). Далее нужно дать
% конструкцию компонент сильной связности как классов эквивалентности
% по отношению обоюдной достижимости на вершинах.

\emph{Ориентированный граф}~--- конечное множество вершин, некоторые из~которых
соединены стрелками.
В~ориентированном графе запрещены кратные стрелки (даже в~разных направлениях)
и~петли, если не~оговорено иное.
Ориентированный граф \emph{полный}, если любая пара его вершин соединена
единственной стрелкой.
Ориентированный граф называется \emph{сильно связным}, если из~любой его
вершины можно добраться до~любой другой по~стрелкам.

\begin{problems}

\item
Докажите, что компоненты сильной связности полного ориентированного графа можно
пронумеровать так, чтобы стрелки между компонентами вели из~компонент с~меньшим
номером в~компоненты с~б\'{о}льшим.

\item
Любые два города Табулистана соединены дорогой с~односторонним движением.
В~стране транспортные проблемы: нет сильной связности.
Докажите, что можно раздать все города республиканцам и~демократам так, чтобы
обе партии получили хотя~бы по~одному городу и~чтобы никакая дорога не~вела
из~города республиканцев в~город демократов.

\item
В~условиях предыдущей задачи, если городов хотя бы три, на~каком наименьшем
числе дорог президенту Табулистана надо изменить направление движения, чтобы
решить транспортные проблемы and make Tabulistan great again?

\item
Докажите, что в~полном ориентированном сильно связном графе на~$n \geq 3$
вершинах через каждую его вершину проходит
\\
\subproblem простой цикл длины $3$;
\\
\subproblem простой цикл любой длины $k$, где $3 \leq k \leq n$.

\item
В~стране 1001 город, любые два города соединены дорогой с~односторонним
движением.
Из~каждого города выходит ровно 500 дорог, в~каждый город входит ровно 500
дорог.
От~страны отделилась независимая республика, в~которую вошли 668 городов.
Докажите, что из~любого города этой республики можно добраться до~любого
другого, не~выезжая за~пределы республики.

\item
Докажите, что в~полном ориентированном графе на~$n \geq 7$ вершинах всегда
найдётся вершина, инвертированием всех стрелок в~которой можно добиться того,
чтобы граф стал сильно связным.

\item
Докажите, что в~полном ориентированном графе на~$n$ вершинах ($n \geq 6$)
существует такой гамильтонов путь $v_1 \to v_2 \to v_3 \to \ldots \to v_{n}$,
что $v_{1} \to v_{n}$, если
\\
\subproblem $n$~--- чётное;
\qquad
\subproblem $n$~--- нечётное.

% С~трен. олимпиады весны-2017.

% Про 2n - 3 и~про гамильтонов цикл давал уже.

\end{problems}


\clearpage\resetproblem \begingroup % \jeolmheader
    \def\jeolmdate{2 июня 2018 г., пара 1}%
    \def\jeolmgroupname{Группа 9-2}%
    \def\jeolmauthors{Афризонов Д., Коваленко К., Тихонов~Ю.}%
\jeolmheader \endgroup

\worksheet*{Ориентированные графы}

% Мне кажется, что тема занятия для второй группы понятие класса
% эквивалентности должно быть новым. По крайней мере на кружке по геометрии с
% векторами была беда совсем ровно потому, что они не понимали, что это такое.
% Лучше им сначала рассказать, и только потом дать листик.

% В начале занятия предполагается рассказ про отношение эквивалентности:
% определение, разбиение на классы эквивалентности + несколько примеров (кольцо
% остатков по модулю, векторы, компоненты связности). Далее нужно дать
% конструкцию компонент сильной связности как классов эквивалентности
% по отношению обоюдной достижимости на вершинах.

\emph{Ориентированный граф}~--- конечное множество вершин, некоторые из~которых
соединены стрелками.
В~ориентированном графе запрещены кратные стрелки (даже в~разных направлениях)
и~петли, если не~оговорено иное.
Ориентированный граф \emph{полный}, если любая пара его вершин соединена
единственной стрелкой.
Ориентированный граф называется \emph{сильно связным}, если из~любой его
вершины можно добраться до~любой другой по~стрелкам.

\begin{problems}

\item
Обозначим через $A_{1}$, $A_{2}$, \ldots, $A_{k}$ компоненты сильной связности
ориентированного графа~$G$.
Введём вспомогательный ориентированный граф~$H$ с~вершинами
$a_{1}$, $a_{2}$, \ldots, $a_{k}$;
проведём стрелку из~$a_{i}$ в~$a_{j}$ в~графе~$H$, если существует хотя бы одна
стрелка из~$A_{i}$ в~$A_{j}$ в~графе~$G$.
\\
\subproblem
Докажите, что в~графе~$H$ нет циклов и кратных стрелок.
\\
\subproblem
Докажите, что если $G$ был полным ориентированным графом, то вершины графа~$H$
можно пронумеровать так, чтобы все стрелки в~графе~$H$ вели из вершины
с~меньшим номером в~вершину с~б\'{о}льшим.
\\
Граф~$H$ называется \emph{конденсацией} ориентированного графа~$G$.

\item
Любые два города Табулистана соединены дорогой с~односторонним движением.
В~стране транспортные проблемы: нет сильной связности.
Докажите, что можно раздать все города республиканцам и~демократам так, чтобы
обе партии получили хотя~бы по~одному городу и~чтобы никакая дорога не~вела
из~города республиканцев в~город демократов.

\item
В~условиях предыдущей задачи, если городов хотя бы три, на~каком наименьшем
числе дорог президенту Табулистана надо изменить направление движения, чтобы
решить транспортные проблемы and make Tabulistan great again?

\item
Докажите, что в~полном ориентированном сильно связном графе на~$n \geq 3$
вершинах через каждую его вершину проходит
\\
\subproblem простой цикл длины $3$;
\\
\subproblem простой цикл любой длины $k$, где $3 \leq k \leq n$.

\item
В~стране 1001 город, любые два города соединены дорогой с~односторонним
движением.
Из~каждого города выходит ровно 500 дорог, в~каждый город входит ровно 500
дорог.
От~страны отделилась независимая республика, в~которую вошли 668 городов.
Докажите, что из~любого города этой республики можно добраться до~любого
другого, не~выезжая за~пределы республики.

\item
Докажите, что в~полном ориентированном графе на~$n \geq 7$ вершинах всегда
найдётся вершина, инвертированием всех стрелок в~которой можно добиться того,
чтобы граф стал сильно связным.

\item
Докажите, что в~полном ориентированном графе на~$n$ вершинах ($n \geq 6$)
существует такой гамильтонов путь $v_1 \to v_2 \to v_3 \to \ldots \to v_{n}$,
что $v_{1} \to v_{n}$, если
\\
\subproblem $n$~--- чётное;
\qquad
\subproblem $n$~--- нечётное.

% С трен. олимпиады весны-2017.

% Про 2n - 3 и про гамильтонов цикл давал уже.

\end{problems}


\clearpage\resetproblem \begingroup % \jeolmheader
    \def\jeolmdate{1 июня 2018 г., пара 3}%
    \def\jeolmgroupname{Группа 9-3}%
    \def\jeolmauthors{Кушнир А., Тихонов~Ю.}%
\jeolmheader \endgroup

\worksheet*{Ориентированные графы}

% В начале занятия предполагается рассказ про отношение эквивалентности:
% определение, разбиение на классы эквивалентности + несколько примеров (кольцо
% остатков по модулю, векторы, компоненты связности). Далее нужно дать
% конструкцию компонент сильной связности как классов эквивалентности
% по отношению обоюдной достижимости на вершинах.

\emph{Ориентированный граф}~--- конечное множество вершин, некоторые из~которых
соединены стрелками.
В~ориентированном графе запрещены кратные стрелки (даже в~разных направлениях)
и~петли, если не~оговорено иное.
Ориентированный граф \emph{полный,} если любая пара его вершин соединена
единственной стрелкой.
Ориентированный граф называется \emph{сильно связным,} если из~любой его
вершины можно добраться до~любой другой по~стрелкам.

\begin{problems}

\item
Докажите, что в~любом полном ориентированном графе существует путь, проходящий
по~всем вершинам ровно по~одному разу \emph{(гамильтонов путь)}.

\item
Обозначим через $A_{1}$, $A_{2}$, \ldots, $A_{k}$ компоненты сильной связности
ориентированного графа~$G$.
Введём вспомогательный ориентированный граф~$H$ с~вершинами
$a_{1}$, $a_{2}$, \ldots, $a_{k}$;
проведём стрелку из~$a_{i}$ в~$a_{j}$ в~графе~$H$, если существует хотя бы одна
стрелка из~$A_{i}$ в~$A_{j}$ в~графе~$G$.
\\
\subproblem
Докажите, что в~графе~$H$ нет кратных стрелок (т.\,е. что между любыми двумя
компонентами графа~$G$ рёбра идут только в~одном направлении).
\\
\subproblem
Докажите, что в~графе~$H$ нет циклов.
\\
\subproblem
Докажите, что если $G$ был полным ориентированным графом, то вершины графа~$H$
можно пронумеровать так, чтобы все стрелки в~графе~$H$ вели из вершины
с~меньшим номером в~вершину с~б\'{о}льшим.
\par
Граф~$H$ называется \emph{конденсацией} ориентированного графа~$G$.

\item
Про ориентированный граф известно, что в~нём не~существует маршрута,
проходящего по~всем вершинами
(даже если в~некоторые вершины заходить по~нескольку раз).
Докажите, что вершины графа можно раскрасить в~красный и~синий цвет так, чтобы
оба цвета присутствовали и~чтобы никакая стрелка не~вела из~красной вершины
в~синюю.

\item
Даны $n$~точек, двое играют в~игру.
За~один ход игрок может соединить стрелкой две точки, не~соединённые ранее.
Запрещено оставлять после своего хода сильно связный граф.
Проигрывает тот, кто не~может сделать ход.
Изначально никаких стрелок не~было.
Кто выигрывает: начинающий или его соперник?

\item
Докажите, что в~любом сильно связном полном ориентированном графе на~$n \geq 3$
вершинах существует цикл, проходящий по~всем вершинам ровно по~одному разу
\emph{(гамильтонов цикл)}.

\item
В~стране $1001$ город, любые два города соединены дорогой с~односторонним
движением.
Из~каждого города выходит ровно $500$ дорог, в~каждый город входит ровно
$500$ дорог.
От~страны отделилась независимая республика, в~которую вошли $668$ городов.
Докажите, что из~любого города этой республики можно добраться до~любого
другого, не~выезжая за~пределы республики.

\end{problems}


\clearpage
\resetproblem \begingroup % \jeolmheader
    \def\jeolmdate{2 июня 2018 г., пара 2}%
    \def\jeolmgroupname{Группа 9-3}%
    \def\jeolmauthors{Афризонов Д., Тихонов~Ю.}%
\jeolmheader \endgroup

\worksheet*{Ориентированные графы, дополнительные задачи}

% По плану на втором занятии группы 9-3 дорешивание первого листик + разбор
% на второй половине занятия.
% Эту добавку надо давать ТОЛЬКО в случае, когда школьник решил ВЕСЬ первый
% листик в самой паре.
% Если таких не нашлось, то не давать вообще.
% У других групп на втором занятии, вероятно, появится новый листик.

% Про разбор.
% Программа минимум — задачи 2 и 3.
% Постараться еще в ходе разбора отложить у них в голове структуру компонент
% сильной связности.
% При разборе задачи 3 надо продемонстрировать тем, кто не понял и не верит,
% в какой степени при выделении компонент задачка может рассыпаться.
% По остальным задачам не лишним будет намекнуть, что произойдет то же самое.

\setproblem{6}

\begin{problems}

\item
Докажите, что в~полном ориентированном сильно связном графе на~$n \geq 3$
вершинах через каждую его вершину проходит
\\
\subproblem простой цикл длины $3$;
\\
\subproblem простой цикл любой длины $k$, где $3 \leq k \leq n$.

\item
Докажите, что в~полном ориентированном графе на~$n \geq 7$ вершинах всегда
найдется вершина, инвертированием всех стрелок в~которой можно добиться того,
чтобы граф стал сильно связным.

\end{problems}

\vfill
\resetproblem \begingroup % \jeolmheader
    \def\jeolmdate{2 июня 2018 г., пара 2}%
    \def\jeolmgroupname{Группа 9-3}%
    \def\jeolmauthors{Афризонов Д., Тихонов~Ю.}%
\jeolmheader \endgroup

\worksheet*{Ориентированные графы, дополнительные задачи}

% По плану на втором занятии группы 9-3 дорешивание первого листик + разбор
% на второй половине занятия.
% Эту добавку надо давать ТОЛЬКО в случае, когда школьник решил ВЕСЬ первый
% листик в самой паре.
% Если таких не нашлось, то не давать вообще.
% У других групп на втором занятии, вероятно, появится новый листик.

% Про разбор.
% Программа минимум — задачи 2 и 3.
% Постараться еще в ходе разбора отложить у них в голове структуру компонент
% сильной связности.
% При разборе задачи 3 надо продемонстрировать тем, кто не понял и не верит,
% в какой степени при выделении компонент задачка может рассыпаться.
% По остальным задачам не лишним будет намекнуть, что произойдет то же самое.

\setproblem{6}

\begin{problems}

\item
Докажите, что в~полном ориентированном сильно связном графе на~$n \geq 3$
вершинах через каждую его вершину проходит
\\
\subproblem простой цикл длины $3$;
\\
\subproblem простой цикл любой длины $k$, где $3 \leq k \leq n$.

\item
Докажите, что в~полном ориентированном графе на~$n \geq 7$ вершинах всегда
найдется вершина, инвертированием всех стрелок в~которой можно добиться того,
чтобы граф стал сильно связным.

\end{problems}

\vfill

\clearpage\input{source/combinatorics/04-variation-r1.tex}
\clearpage\resetproblem \begingroup % \jeolmheader
    \def\jeolmdate{3 июня 2018 г., пара 1}%
    \def\jeolmgroupname{Группа 9-2}%
    \def\jeolmauthors{Афризонов Д., Кушнир А., Тихонов~Ю.}%
\jeolmheader \endgroup

\worksheet*{Вариация}

\begin{problems}

\item
На~отрезке~$AB$ отмечено $2n$ различных точек, симметричных относительно
середины $AB$.
При этом $n$ из~них покрашены в~красный цвет, оставшиеся $n$~--- в~синий.
Докажите, что сумма расстояний от~точки~$A$ до~красных точек равна сумме
расстояний от~точки~$B$ до~синих точек.

\item
Сумма нескольких натуральных чисел равна $2017$.
Найдите максимально возможное их произведение.

%\item
%Имеется три кучи камней.
%Сизиф таскает по~одному камню из~кучи в~кучу.
%За~каждое перетаскивание он получает от~Зевса количество монет, равное разности
%числа камней в~куче, в~которую он кладет камень, и~числа камней в~куче,
%из~которой он берет камень
%(сам перетаскиваемый камень при этом не~учитывается).
%Если указанная разность отрицательна, то~Сизиф возвращает Зевсу соответствующую
%сумму.
%(Если Сизиф не~может расплатиться, то~великодушный Зевс позволяет ему совершать
%перетаскивание в~долг.)
%В~некоторый момент оказалось, что все камни лежат в~тех~же кучах, в~которых
%лежали первоначально.
%Каков наибольший суммарный заработок Сизифа на~этот момент?
%% Прибережем до~лучших времен.

\item
В~однокруговом турнире по~теннису участвовало $2n+1$ человек:
$p_{1}, p_{2}, \ldots, p_{2n+1}$
(каждый сыграл с~каждым ровно один раз, ничьих не~бывает).
Обозначим через~$w_{i}$ число побед игрока~$p_{i}$.
Найдите максимум и~минимум (в~зависимости от~$n$) величины
$w_{1}^2 + w_{2}^2 + \ldots + w_{2n+1}^2$.

%\item
%На~прямой отмечены $2n$ различных точек, при этом $n$ из~них покрашены
%в~красный цвет, остальные $n$~--- в~синий.
%Докажите, что сумма попарных расстояний между точками одного цвета
%не~превосходит суммы попарных расстояний между точками разного цвета.

\item
По~окружности расставлены несколько положительных чисел, не~превосходящих
единицы.
Докажите, что окружность можно разбить на~$2018$ дуг так, чтобы суммы чисел
на~соседних дугах отличались не~более чем на~один.
Если чисел на~дуге нет, то~сумма чисел на~дуге считается равной нулю.

\item
Председателю дачного кооператива необходимо распределить $49$ квадратных
участков (в~виде квадрата $7 \times 7$) среди $49$~дачников.
Каждый дачник враждует не~более чем с~$6$ другими.
Докажите, что это можно сделать так, чтобы никакие два враждующих дачника
не~получили соседние по~стороне участки.

\item
В~таблице $n \times m$ расставлены действительные числа так, что сумма чисел
в~каждом столбце и~каждой строке целая.
Докажите, что каждое число в~таблице можно заменить на~его верхнюю или нижнюю
целую часть так, чтобы сумма чисел в~каждом столбце и~каждой строке
не~изменилась.

\item
\emph{Хромой ладьей} назовем ладью, которая за~один ход может сдвинуться только
на~одну клетку.
Хромая ладья за~$64$ хода обошла все клетки шахматной доски и~вернулась
на~исходную клетку.
Докажите, что число ее ходов по~горизонтали не~равно числу ходов по~вертикали.

% про яблоки и бананы с зоналки задача еще подходит, но ее давал сильной группе
% на весенних сборах.
% Есть еще про додекаэдр от Кушнира, тоже давали уже.

\end{problems}


\clearpage\input{source/combinatorics/03-variation-r3.tex}

\clearpage\input{source/combinatorics/05-infinity-r1.tex}
\clearpage\input{source/combinatorics/05-infinity-r2.tex}
\clearpage\resetproblem \begingroup % \jeolmheader
    \def\jeolmdate{7 июня 2018 г., пара 1}%
    \def\jeolmgroupname{Гексаэдры}% 9-3
    \def\jeolmauthors{Коваленко К., Кушнир А., Тихонов Ю.}%
\jeolmheader \endgroup

\worksheet*{Сколь угодно много и бесконечно много}

\begin{problems}

\item Натуральные числа раскрасили в два цвета. Обязательно ли существует одноцветная бесконечная арифметическая прогрессия? 

\item Зафиксировано натуральное число $n$. Будем говорить, что числовая строка $(x_1, x_2, \ldots, x_n)$ \emph{лексикографически больше} числовой строки $(y_1, y_2, \ldots, y_n)$, если есть такой $i \in \{ 1, 2, \ldots, n \}$, что $x_1 = y_1, x_2 = y_2, \ldots, x_{i - 1} = y_{i - 1}, x_i > y_i$.\\
\claim{Лемма} Докажите, что не существует бесконечной лексикографически убывающей последовательности строк длины $n$, состоящих из натуральных чисел.

\item В Табулистане все купюры местной валюты обладают целочисленными номиналом. В местном банке готовы принять любую купюру и выдать вместо неё любое количество купюр меньшего номинала. У барона Мюнхгаузена есть купюра достоинством $100$, и он утверждает, что сможет жить на эти деньги бесконечно долго, расходуя при этом не менее $1$ единицы валюты в день. Стоит ли ему доверять?

\item \claim{Лемма Кёнига} Докажите, что в бесконечном дереве, степень каждой вершины которого конечна, найдётся бесконечный простой путь.

\item В стране роботов любые конечные или бесконечные последовательности из $0$~и~$1$ называются \emph{словами}. Некоторые конечные слова объявлены \emph{матерными}. Слово называется \emph{цензурным}, если оно не содержит матерных подслов.\\
\subproblem Известно, что существуют сколь угодно длинные конечные цензурные слова. Докажите, что существует бесконечное цензурное слово.\\
\subproblem Известно, что матерных слов конечное число и существует бесконечное цензурное слово. Докажите, что существует бесконечное периодическое цензурное слово.
%\subproblem Останется ли верным утверждение предыдущего пункта, если снять ограничение на конечность множества матерных слов?

\item Допустим, что любую конечную карту можно правильным образом раскрасить в 4 цвета. Докажите, что тогда и бесконечную карту тоже можно раскрасить в 4 цвета.

\item Вася закрашивает точки координатной плоскости с натуральными координатами. За одну операцию он выбирает ешё не закрашенную точку ($m, n$) и закрашивает все точки $(x, y)$, у которых $x \geq m$ и $y \geq n$. Может ли Вася проворачивать свои операции бесконечно долго?

\item Сборная Табулистана с $2000$ года ежегодно участвует в финале всероссийской олимпиады школьников по математике, которая проводится по параллелям 9,~10,~11 класса. Год $n$ называется \emph{провальным}, если для каждого предыдущего года $m$ существует параллель, по которой в год $n$ сборная Табулистана взяла меньше дипломов, чем в год $m$. Могут ли все годы участия, начиная с $2001$, быть провальными?

% \item Есть несколько эталонных квадратов $1 \times 1$, стороны которых раскрашены в какие-то цвета и параллельны осям координат. Разрешается полученные параллельным переносом копии этих квадратов выкладывать на плоскость, причём соседние квадраты должны задавать одинаковую раскраску их общей стороны. Известно, что исходный набор квадратиков таков, из его копий можно выложить сколько угодно большой квадрат $n \times n$. Докажите, что копиями квадратов набора можно замостить всю плоскость.
% Оставим на повторение.

% \item Все натуральные числа покрасили в~несколько цветов. Докажите, что найдется цвет такой, что для любого натурального~$n$ бесконечно много чисел этого цвета делится на~$n$. 
% Не совсем в тему; простой конструктив. Перемножим все плохие числа для всех цветов.

%\item Натуральные числа покрашены в~синий и~зеленый цвета, причем чисел каждого цвета бесконечно много. Докажите, что можно выбрать миллион синих и~миллион зеленых чисел так, что сумма выбранных синих равна сумме выбранных зеленых. 

%\item Имеется бесконечная шахматная доска. Из~нее выкинули все клетки, у~которых обе координаты отрицательны или обе координаты больше тысячи. Можно~ли обойти такую фигуру (побывав в~каждой клетке ровно по~одному разу) конем?
% Вообще не в тему. Асимптотика.

% \item Каждая точка плоскости, имеющая целочисленные координаты, раскрашена в~один из~$n$ цветов. Докажите, что найдется прямоугольник с~вершинами в~точках одного цвета. 

% \item Каждая точка плоскости, имеющая целочисленные координаты, раскрашена в один из $n$~цветов. Докажите, что для любого $N$ найдется цвет~$X$ такой, что найдется точка цвета~$X$, с которой на одной горизонтали бесконечно много точек цвета~$X$, а на одной вертикали не меньше $N$ точек цвета~$X$. 
% Не нравится она мне.

% \ii За~дядькой Черномором выстроилось чередой бесконечное число богатырей. Докажите, что он может приказать части из~них выйти из~строя так, чтобы в~строю осталось бесконечно много богатырей и~все они стояли по~росту (не~обязательно в~порядке убывания роста). 

% \item Имеется возрастающая последовательность натуральных чисел $a_{1} < a_{2} < \ldots < a_{n} < \ldots$, где для всякого $k$ выполнено $a_{k+1} - a_{k} < 10^6$. Докажите, что найдутся различные $i$ и~$j$ такие, что $a_i$ делится на~$a_j$. 
% Крутая задача, но не совсем по теме.

%\item Натуральные числа покрасили в~$100$ цветов. Докажите, что найдется цвет~$X$ такой, что есть бесконечно много пар чисел цвета~$X$ на~расстоянии меньше $200$.
% Следующая --- примерно про то же.

% \item Натуральные числа покрашены в~несколько цветов. Докажите, что найдутся цвет~$X$ и~число~$m$ такие, что для любого натурального $k$ существуют натуральные числа $a_{1} < a_{2} < \ldots < a_{k}$ цвета~$X$, где $a_{i+1} - a_{i} < m$ при всех $1 \leq i < k$.

% \item Мудрецы выстроены в бесконечный ряд. На каждом --- колпак одного из двух цветов. Каждый мудрец знает свой номер в ряду и видит всех мудрецов с большими номерами. Мудрецов по команде должны одновременно назвать предполагаемый цвет своего колпака. Докажите, что мудрецы могут заранее договориться о стратегии так, чтобы при любой расстановке колпаков не угадало цвет своего колпака лишь конечное число мудрецов.
% Задача от Саши Фёдорова.

\end{problems}

\clearpage\resetproblem \begingroup % \jeolmheader
    \def\jeolmdate{8 июня 2018 г., пара 1}%
    \def\jeolmgroupname{Группа 9-1}%
    \def\jeolmauthors{Кушнир А., Тихонов~Ю.}%
\jeolmheader \endgroup

\worksheet*{Дополнительные задачи по комбинаторике}

\begin{problems}

\item Последовательность $a_n$ натуральных чисел называется \emph{полиномиальной}, если существует многочлен $P(x)$ с целыми коэффициентами такой, что $a_n = P(n)$ при всех натуральных $n$. Можно ли раскрасить натуральные числа в два цвета так, чтобы не было одноцветных полиномиальных последовательностей?

\item Стрелки полного ориентированного графа раскрашены в два цвета. Докажите, что в нём существует вершина, от которой можно добраться до любой другой вершины по некоторому монохромному пути одного из двух цветов.

\item В $100$ ящиках лежат яблоки, апельсины и бананы. Докажите, что можно так выбрать $51$ ящик, что в них окажется не менее половины всех яблок, не менее половины всех апельсинов и не менее половины всех бананов.

\end{problems}\vfill
\resetproblem \begingroup % \jeolmheader
    \def\jeolmdate{8 июня 2018 г., пара 1}%
    \def\jeolmgroupname{Группа 9-1}%
    \def\jeolmauthors{Кушнир А., Тихонов~Ю.}%
\jeolmheader \endgroup

\worksheet*{Дополнительные задачи по комбинаторике}

\begin{problems}

\item Последовательность $a_n$ натуральных чисел называется \emph{полиномиальной}, если существует многочлен $P(x)$ с целыми коэффициентами такой, что $a_n = P(n)$ при всех натуральных $n$. Можно ли раскрасить натуральные числа в два цвета так, чтобы не было одноцветных полиномиальных последовательностей?

\item Стрелки полного ориентированного графа раскрашены в два цвета. Докажите, что в нём существует вершина, от которой можно добраться до любой другой вершины по некоторому монохромному пути одного из двух цветов.

\item В $100$ ящиках лежат яблоки, апельсины и бананы. Докажите, что можно так выбрать $51$ ящик, что в них окажется не менее половины всех яблок, не менее половины всех апельсинов и не менее половины всех бананов.

\end{problems}\vfill

\clearpage\resetproblem \begingroup % \jeolmheader
    \def\jeolmdate{9 июня 2018 г., пара 1}%
    \def\jeolmgroupname{Группа 9-1}%
    \def\jeolmauthors{Афризонов Д., Кушнир А., Тихонов~Ю.}%
\jeolmheader \endgroup

\worksheet*{Решётки}

\begin{problems}

\item
В вершинах некоторого квадрата сидит по одному кузнечику. Каждую минуту ровно один кузнечик перепрыгивает центрально симметрично через какого-то другого.
\\
\subproblem Докажите, что если кузнечики вновь оказались в вершинах изначального квадрата, то каждый вернулся именно на свою стартовую позицию.
\\
\subproblem Могут ли кузнечики когда-нибудь оказаться в вершинах большего квадрата, чем изначальный?
% \item В трёх вершинах квадрата сидят кузнечики. Каждый год один из кузнечиков перепрыгивает через одного из двух других (т.\,е. отражается центрально симметрично). В конце они вновь оказываются в каких-то трёх вершинах исходного квадрата. Докажите, что каждый кузнечик сидит в своей стартовой вершине.

% \item Дан выпуклый $31500$-угольник, вершины которого располагаются в узлах целочисленной решётки. Докажите, что у него найдётся сторона длины по крайней мере $98$.

\item На координатной плоскости расположена фигура площади больше $1$. Докажите, что в ней можно отметить две различные точки так, чтобы вектор, их соединяющий, имел целые координаты.

\item На плоскости расположено несколько квадратов со стороной $1$, параллельных осям координат. Докажите, что в плоскость можно вбить несколько гвоздей так, чтобы каждый квадрат был прибит ровно одним гвоздем.

% \item Два вектора $\mathbf{u}$ и $\mathbf{v}$ с целыми координатами таковы, что все целые точки плоскости можно выразить их целочисленными линейными комбинациями (т.\,е. выражениями вида $n \mathbf{u} + m \mathbf{v}$, где $m$ и $n$ --- целые числа). Докажите, что параллелограмм, натянутый на $\mathbf{u}$ и $\mathbf{v}$, имеет площадь $1$.

\item Внутри правильного треугольника отмечена точка. Её отразили несколько раз относительно сторон треугольника (в случайном порядке), и она опять попала внутрь треугольника. Докажите, что она вернулась на исходное место.

\item Существует ли на клетчатой плоскости замкнутая ломаная с нечётным числом звеньев одинаковой длины, все вершины которой лежат в узлах целочисленной решётки?

\item Время жизни марсиан составляет в точности $100$ лет. Каждый марсианин рождается ровно в момент начала какого-то года. За всё время жизни вселенной существовало нечётное число марсиан. Докажите, что можно выделить по крайней мере $100$ лет (не обязательно подряд идущих), когда жило нечётное число марсиан.

% \item (\emph{Теорема Минковского}) На координатной плоскости дана выпуклая ограниченная фигура площади больше, чем $4$, симметричная относительно начала координат. Докажите, что она содержит целую точку, отличную от начала координат.
 % Фигура на плоскости называется \emph{выпуклой}, если вместе с каждой парой своих точек $A$ и $B$ она целиком содержит отрезок $AB$.

% \item На плоскости расположено $2015$ квадратов со стороной $1$, параллельных осям координат. Докажите, что множество точек плоскости, покрытых нечётным количеством таких квадратов, имеет площадь не менее $1$.

\item На клетчатой плоскости дан многоугольник, вершины которого лежат в узлах целочисленной решётки. Каждую клетку плоскости разрезали красными диагоналями. В результате плоскость оказалась разбита на квадратики с красной границей, которые покрасили в шахматном порядке в чёрный и белый цвет. Докажите, что площадь белой части исходного многоугольника равна площади чёрной части.

\item Рассмотрим на плоскости множество точек с вещественными координатами $(x, y)$ таких, что неравенство $mx + ny \geqslant \frac{1}{2} \cdot (m^2 + n^2)$ имеет ровно $2018$ целочисленных решений $(m, n)$. Найдите площадь этого множества.

% \item На координатной плоскости отмечены некоторые целые точки. Известно, что внутри любого круга радиуса $2015$ хотя бы одна точка отмечена. Докажите, что существует окружность, проходящая хотя бы через четыре отмеченные точки.

\end{problems}
\clearpage\input{source/combinatorics/09-lattices-r2.tex}

\clearpage\resetproblem \begingroup % \jeolmheader
    \def\jeolmdate{09 июня 2018 г., пара 2}%
    \def\jeolmgroupname{9-3}%
    \def\jeolmauthors{Соколов~А.\,А.}%
\jeolmheader \endgroup

\worksheet*{Теорема Паскаля}

\claim{Теорема Паскаля} Даны шесть точек $A, B, C, D, E, F$ на одной окружности. Тогда пересечения прямых $AB$ и $DE$, $BC$ и $EF$, $CD$ и $FA$ лежат на одной прямой.


\begin{problems}

\item В окружность вписан шестиугольник $ABCDEF$. Отрезок $AC$ пересекается с отрезком $BF$ в точке $X$, $BE$ с $AD$ -- в точке $Y$ и $CE$ и $DF$ в точке $Z$. Докажите, что точки $X$, $Y$ и $Z$ лежат на одной прямой\footnote{Попробуйте найти на картинке два подобных треугольника $ABY$ и $EDY$ с изогонально сопряжёнными точками $X$ и $Z$}.

%\item  \emph{(Теорема Паскаля для треугольника)}  Докажите, что в неравнобедренном треугольнике точки пересечения касательных к описанной окружности, восстановленных в вершинах, с противоположными сторонами лежат на одной прямой.
\item Окружность, проходящая через вершины $A$ и $D$ основания трапеции $ABCD$, пересекает боковые стороны $AB$, $CD$ в точках $P$, $Q$, а диагонали -- в точках $E, F$. Докажите, что прямые $BC, PQ, EF$ пересекаются в одной точке.

%\item Даны треугольник $ABC$ и некоторая точка $T$. Пусть $P$ и $Q$ -- основания перпендикуляров, опущенных из точки $T$ на прямые $AB$ и $AC$ соответственно, a $R$ и $S$ -- основания перпендикуляров, опущенных из точки $A$ на прямые $TC$ и $TB$ соответственно. Докажите, что точка пересечения прямых $PR$ и $QS$ лежит на прямой $BC$.

\item \emph{(Теорема Паскаля для четырёхугольника)} Докажите, что прямая, соединяющая точки пересечения пар противоположных сторон вписанного в окружность четырёхугольника, совпадает с прямой, соединяющей точки пересечения пар касательных к этой окружности, восставленных в противоположных вершинах.


\item Внутри треугольника $ABC$ отмечена точка $P$. Прямые $AP, BP, CP$ вторично пересекают описанную окружность треугольника $ABC$ в точках $A_1, B_1, C_1$ соответственно. Докажите, что главные диагонали шестиугольника, полученного пересечением треугольников $ABC$ и $A_1B_1C_1$ пересекаются в точке $P$.


\item Четырехугольник $ABCD$ вписан в окружность с центром $O$. Точка $X$ такова, что $\angle BAX = \angle CDX = 90^{\circ}$. Докажите, что точка пересечения диагоналей четырехугольника $ABCD$ лежит на прямой $XO$.



\item Пусть $A'$ -- точка, диаметрально противоположная точке $A$ в описанной окружности треугольника $ABC$ с центром $O$. Касательная к описанной окружности в точке $A'$ пересекает прямую $BC$ в точке $X$. Прямая $OX$ пересекает стороны $AB$ и $AC$ в точках $M$ и $N$. Докажите, что $OM = ON$.

\item На сторонах $AB$ и $AC$ остроугольного треугольника $ABC$ выбрали соответственно точки $M$ и $N$ так, что отрезок $MN$ проходит через центр $O$ описанной окружности треугольника $ABC$. Пусть $P$ и $Q$ -- середины отрезков $CM$ и $BN$. Докажите, что $\angle POQ = \angle BAC$.
% 3ка с летних сборов 2014

% \item Треугольники $ABC$ и $A'B'C'$ вписаны в одну и ту же окружность, и их пересечением является шестиугольник. Докажите, что главные диагонали шестиугольника пересекаются в одной точке.


%\item На стороне $AB$ треугольника $ABC$ взята точка $D$. В угол $ADC$ вписана окружность, касающаяся изнутри описанной окружности треугольника $ACD$, а в угол $BDC$ -- окружность, касающаяся изнутри описанной окружности треугольника $BCD$. Оказалось, что эти окружности касаются отрезка $CD$ в одной и той же точке $X$. Докажите, что перпендикуляр, опущенный из $X$ на $AB$, проходит через центр вписанной окружности треугольника $ABC$.




%\item В  треугольнике $ABC$ проведены  высоты $AA_1$ и $BB_1$ и  биссектрисы $AA_2$ и $BB_2$; вписанная  окружность  касается  сторон $BC$ и $AC$ в  точках $A_3$ и $B_3$.  Докажите,  чтопрямые $A_1B_1, A_2B_2, A_3B_3$ пересекаются  в  одной  точке или параллельны.

%\item  Равносторонний треугольник $ABC$ вписан в окружность $\Omega$ и описан вокруг окружности $\omega$. На сторонах $AC$ и $AB$ выбраны точки $P$ и $Q$ соответственно так, что отрезок $PQ$ проходит через центр треугольника $ABC$. Окружности $\Gamma_b$ и $\Gamma_c$ построены на отрезках $BP$ и $CQ$ как на диаметрах. Докажите, что окружности $\Gamma_b$ и $\Gamma_c$ пересекаются в двух точках, одна из которых лежит на $\Omega$, а другая -- на $\omega$.

\end{problems}

% }}}1

% vim: set foldmethod=marker :%%%%%%%%%%%%%%%%%%%%%%%%%%%%%%%%%%%%%%%%%%%%%%%%%%
